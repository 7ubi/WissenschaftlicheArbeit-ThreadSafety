\section{Statische Race Detection}

Diese Sektion beschäftigt sich mit Statischer Race Detection. Diese funktioniert in den meisten Algorithmen, wie zum Beispiel Relay \cite[vgl.][208]{relay} oder RacerD \cite[vgl.][57]{nikos_2019}, mittels der Analyse von Locks in einem gegebenen Programm. Diese Analyse ist ähnlich, wie die Dynamische Race Detection durch Lockset Analyse. Die Sektion wird sich auf den von Facebook genutzten RacerD Algorithmus \cite[vgl.][2]{racerd} fokussieren und diesen erklären. 

\subsection*{RacerD}

RacerD analysiert alle nicht private Methoden von Klassen eines gegebenen Programms. Dabei müssen zwei Fragen beantwortet werden, für ein Paar Zugriffe einer Klasse \cite[vgl.][7]{racerd}:

\begin{enumerate}
	\item Sind die Zugriff auf die selbe Adresse?
	\item Sind die Zugriffe gleichzeitig?
\end{enumerate}

Um diese Fragen zu beantworten gibt es in RacerD die Locks, Threads, Ownership und Access Snapshots Domänen \cite[vgl.][7-8]{racerd}. 

\subsubsection*{Locks Domäne}

Die Locks Domäne erkennt, dass kein Data Race entsteht, dadurch, dass das Objekt durch das selbe Lock geschützt wird und deswegen nur ein Thread gleichzeitig drauf zugreifen kann \cite[vgl.][8]{racerd}. RacerD hat als Ziel eine kleine Menge an Data Races zu finden, die mit hohem Vertrauen auch Races sind. Deshalb fokussiert sich RacerD darauf, unsicher Stellen zu finden, die durch kein Lock geschützt sind. Dabei versucht RacerD keine Races zu finden, die entstehen, da das Objekt durch das falsche Lock geschützt sind \cite[vgl.][9]{racerd}. 

\subsubsection*{Threads Domäne}

Die Threads Domäne erkennt, das kein Data Race vorliegt, indem es erfasst, ob ein Stelle nur von einem Thread aufgerufen werden kann \cite[vgl.][8]{racerd}. Um zu festzustellen, ob eine Stelle im Programm mehreren Thread aufgerufen wird $"$concurrent context inferernce$"$ \cite[9]{racerd} genutzt. Dafür gibt es drei abtrakte Zustände.
Gestartet wird im Stadium $"$NoThread$"$. Dieser bedeutet, dass diese Stelle nur auf einem Thread ausgeführt wird \cite[vgl.][9-10]{racerd}.\\
\\
Werden Annotationen, wie \texttt{@ThreadSafe} oder \texttt{@GuardedBy} genutzt, wird in den Zustand $"$AnyThread$"$ übergegangen. In diesem Zustand können Programm Code Stellen, die auf einem Thread laufen, der mit jedem anderen Thread verschachtelt werden kann. Dasselbe gilt auch bei Verwendung des \texttt{synchronized} keyword oder anderen Locks \cite[vgl.][10]{racerd}.\\
\\
Wenn die Annotation \texttt{@UiThread} oder die Methode \texttt{assertMainThread()} benutzt wird ändert sich der Zustand in $"$AnyThreadButMain$"$. Diese Prozeduren laufen auf dem Main Thread und können parallel zu Code in Hintergrund Threads laufen \cite[vgl.][10]{racerd}.

\subsubsection*{Ownership Domäne}

Die Ownership Domäne bezieht sich auf Speicher, welcher nur von einem Thread angesprochen werden kann und dadurch zu keinem Race führen kann. \ref{lst:ownership} gilt als Bespiel hierfür. Die Methode \texttt{ownedAccess()} kann ohne Data Races von mehreren Threads ausgeführt werden. Dies liegt daran, dass die Threads zwar das gleiche ausführen, aber jeder Thread eine neue Instanz \texttt{x} der Klasse \texttt{C} sich erzeugt. Es macht sich die Tatsache zunutze, dass ein neu initialisiertes Objekt über den Befehl \texttt{new} zugewiesen wird \cite[vgl.][10]{racerd}. \\

%\begin{comment}
\begin{lstlisting}[language=Java,frame=tb,caption={Ownership Domäne \cite{racerd}}, label={lst:ownership}, numbers=left, stepnumber=1, captionpos=b, tabsize=4]
void ownedAccess() {
	C x = new C();
	c.x = 42;
}
\end{lstlisting}
%\end{comment}

\subsubsection*{Access Snapshots Domänen}

Die Access Snapshots Domäne nimmt relevante Informationen und speichert sie in einem \texttt{Access Snapshot}, sobald auf Dynamischen Speicher zugegriffen wird \cite[vlg.][11]{racerd}. Gespeichert werden \cite[vgl.][8]{racerd}:
\begin{itemize}
	\item Die Stelle im Speicher, welche aufgerufen wird.
	\item Die Art des Zugriffs, also ob gelesen oder geschrieben wird.
	\item Wie viele Locks gehalten werden.
	\item Ob die Stelle von mehreren Threads ausgeführt wird (Thread Domäne).
	\item Ob die Stelle von nur einem Thread angesprochen werden kann (Ownership Domäne).
\end{itemize}

\subsubsection*{Data Races erkennen}

Um Data Races mit RacerD zu erkennen, wird jedes Paar Access Snapshots miteinander verglichen. Um ein Data Race zu erzeugen müssen beide Access Snapshots auf die selbe Speicherstelle zugreifen und eine Operation auf diese Stelle muss eine Schreiboperation sein. Zudem muss mindestens einer der Zugriffe ohne Lock passieren. Außerdem darf keiner der beiden Access Snapshots nur von einem Thread angesprochen werden. Zuletzt muss der Wert für die Thread Domäne $"$AnyThread$"$ besitzen, also darf mindestens einer der Threads nicht der Main Thread sein \cite[vgl.][8]{racerd}. Wenn ein Race gefunden wurde, wird ein Stacktrace zurückgegeben, sodass man den Bug analysieren kann \cite[vgl.][15]{racerd}.

