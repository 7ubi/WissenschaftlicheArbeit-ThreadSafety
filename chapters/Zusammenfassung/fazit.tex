\section{Fazit}

Ziel der Arbeit war es, den Lesenden zu zeigen, was Thread Safety ist und warum man es braucht. Des Weiteren sollte die Arbeit erläutern, wie man Verstöße erkennt und erkannte Verstöße löst.

Was Thread Safety ist und wie ein Data Race entsteht, wurde in den \hyperref[sec:threadSafety]{Theoretischen Grundlagen} erklärt.


Anschließend wurde darauf eingegangen, wie man durch Dynamische bzw. Statische Race Detection Data Races in einem Programm erkennen kann. Dabei wurde die Dynamische Race Detection zwischen der \acs{HB} Beziehung und dem Lockset Algorithmus unterschieden.

Zuletzt wurde beispielhaft an dem Synchronized Keyword in Java erklärt, wie man die gefunden Data Races beheben kann und welche Probleme, also Deadlocks oder Livelocks, entstehen können, wenn man schlechte Synchronization verwendet. 