Im folgenden Kapitel wird auf die Thematik von Threads eingegangen und was Multithreading ist. Im Anschluss daran wird das Thema Thread Safety behandelt und wie durch Multithreading Probleme enstehen können. 

\section{Threads und Multithreading}\label{sec:threads}

Ein Thread ist ein Programm, welches aus einer Reihe von unabhängigen Befehlen besteht. Die meisten Programme starten mit einem Thread. In Multiprozessor Systemen können diese Threads parallel ausgeführt werden. Programme erstellen im Laufe ihrer Ausführung weitere Threads oder beenden die von ihnen zuvor erstellten Threads. Dieses Konzept wird Multithreading genannt \cite[vgl.][70]{banerjee_theory_2006}.\\
\\
Threads, die zu einem gleichen Prozess gehören, greifen zudem auf denselben Adressraum zu und haben somit die gleichen Variablen und erzeugte Objekte Hier befindet sich das Problem der Thread Safety, da verschiedene Threads zur gleichen Zeit auf die gleichen Ressourcen zugreifen können \cite[vgl.][2]{brian}.