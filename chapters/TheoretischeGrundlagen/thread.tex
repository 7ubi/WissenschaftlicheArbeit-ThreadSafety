\section{Threads und Multithreading}\label{sec:threads}

Die meisten Programme starten mit einem Thread. In Multiprozessor Systemen können diese Threads parallel ausgeführt werden. Programme erstellen im Laufe ihrer Ausführung weitere Threads oder beenden die von ihnen zuvor kreierte Threads. Dieses Konzept wird Multithreading genannt \cite[vgl.][70]{banerjee_theory_2006}.

Threads, die zu einem gleichen Prozess gehören, greifen zudem auf denselben Adressraum zu und haben somit die gleichen Variablen und erzeugte Objekte Hier befindet sich das Problem der Thread Safety, da verschiedene Threads zur gleichen Zeit auf die gleichen Ressourcen zugreifen können \cite[vgl.][2]{brian}.