\chapter{Stand der Forschung}

Zu dem ausgewählten Thema $"$\textbf{Thread Safety}$"$ liegt bereits einige Literatur von beispielsweise \textcite{li_efficient_2019} und \textcite{erickson_effective_nodate} vor. Das Thema $"$\textbf{Thread Safety Probleme überprüfen}$"$ wurde dabei in der Literatur in Dynamische und Statische Race Detection Programme aufgeteilt. Mit Dynamischer Erkennung beschäftigten sich \textcite{li_efficient_2019}, \textcite{erickson_effective_nodate} und \textcite{savage_eraser_nodate}, wohingegen sich \textcite{relay} und \textcite{racerd} mit Statischer Erkennung beschäftigen. \textcite{li_efficient_2019} und \textcite{erickson_effective_nodate} gehen hierbei auf Dynamische Race Detection durch die \ac{HB} Beziehung ein, wohingegen \textcite{savage_eraser_nodate} auf den Lockset Algorithmus eingeht.\\
\\
Mit dem Thema $"$\textbf{wie man diese Probleme nach dem Erkennen dann löst}$"$ beschäftigt sich \textcite{brian} und \textcite{fekete_teaching_nodate}. Beide gehen dazu beispielhaft auf die Programmiersprache Java ein. \textcite{fekete_teaching_nodate} geht jedoch aus der Sicht lehrender Professoren der University of Sydney an das Thema heran und behandelt hierbei wie sie Studierenden das Schreiben von Thread sicheren Klassen beibringen kann.