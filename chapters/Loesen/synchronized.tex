\label{sec:loesen}

Diese Sektion beschäftigt sich damit, wie die zuvor gefundenen Thread Safety Verstöße behoben werden können. Die Erklärung wird Beispielhaft in Java gemacht. Für andere Programmiersprachen gibt es aber ähnliche Methoden, welche man benutzen kann um diese Fehler zu beheben.

\section{Synchronized}

 Java bietet mehrere Optionen zum Lösen der Verstöße. Zum einen bietet Java seit Java 5 die \texttt{java.util.concurrent} Bibliothek, mit der sich der nächste Teil beschäftigt. Zum andern wird das \texttt{synchronized} keyword gestellt, mit dem sich diese Sektion beschäftigt \cite[vgl.][121]{fekete_teaching_nodate}. 

 Um das \texttt{synchronized} keyword zu erläutern wird das Beispiel, aus den \hyperref[sec:threads]{Theoretische Grundlagen}, \ref{lst:threadSafety}. In \ref{lst:threadSafetySolved} wird eine Klasse beschrieben, die dieselbe Funktionalität wie die Klasse in \ref{lst:threadSafety} hat. Der Unterschied zwischen den Klassen ist, dass \ref{lst:threadSafetySolved} Thread sicher ist und somit von mehreren Threads gleichzeitig aufgerufen werden kann, ohne das Problem entstehen \cite[vgl.][5-6]{brian}. 
\\
%\begin{comment}
 \begin{lstlisting}[language=Java,frame=tb,caption={Thread-safe Sequence Generator \cite{brian}}, label={lst:threadSafetySolved}, numbers=left, stepnumber=1, captionpos=b]
public class Sequenz {
	@GuardedBy("this") private int value;

	public synchronized int getNext() {
		return value++;
	}
}
\end{lstlisting}
%\end{comment}

Das \texttt{synchroized} keyword setzt dabei eine Art Schloss, auch Lock genannt, auf das Objekt, sodass nur ein Thread gleichzeitig darauf zugreifen kann. Wenn also die Methode \texttt{getNext()} von einem Objekt der Klasse \texttt{Sequenz} vom Thread A aufgerufen wird und der Thread B dieselbe Methode des gleichen Objekts aufrufen will, muss Thread B solange warten bis A fertig ist und das Schloss auflöst \cite[vgl.][17]{brian}.



% Optimierung von Synchronization?